\chapter{General Introduction}
\label{chap:introduction}%Note this label will be used to refer to the chapter throughout. So if you change the order of chapters it still knows this one is this file, but can call it chapter 1 or 2 or whatever depending on the order. S oti's better than calling it chapter 1.

%\begin{quoteshrink}
 % ``I can mention many moments that were unforgettable and revelatory. But the most single revelatory three minutes was the first time I put on scuba gear and dived on a coral reef. It's just the unbelievable fact that you can move in three dimensions.''
%  \hfill{David Attenborough}
%\end{quoteshrink}

\noindent
The ability to both obtain and avoid becoming food is one of the primary selection pressures driving animal evolution. Predator-prey interactions not only shapes the species directly involved in these interactions but also form the fundamental building blocks of ecosystem structure making them key to our understanding of biological systems as a whole \citep{pimm1984complexity,cohen1990community}. However while these interactions are ubiquitous across diverse ecosystems each predator-prey interaction plays out within its own specific context. For example, even if hunting the same prey one predator may rely on venom to incapacitate prey while another may rely on a strategy of high performance aerobatics to meet the same ends. However amongst these context dependencies commonalities arise across that provide a clear approach to understand how these interactions emerge and effect they have on the systems in which they are embedded.


While the arms race between predators and their prey plays out across a diversity of forms, all players must abide to the fundamental constraints imposed by physics. For example biomechanical and physiological limitations result in flying species remaining relatively small \citep{chatterjee2007aerodynamics,dudley2002mechanisms} while the largest species are invariably found in the oceans \citep{heim2015cope}. It is within these boundaries of physics that evolution trades off between the benefits of investing in traits relating to that species predator-prey interactions and the energetic costs associated with developing such traits. One common way for these trades-offs to present themselves is through there link with both body size and interaction dimensionality.


Since Kleiber fist demonstrated the link between the rate of energy utilised by an animal and its size \citep{kleiber1947body} ecology has utilised this correlation to understand a range of biological processes and principles (refs). More recent formulations have attempted to use the fractal structure of physiological systems as a first principle approach to unify many aspect of biology including physiology, behaviour and ecology \citep{west1997general,brown2004}. Despite the heated debate surrounded the exact nature of this scaling and the fundamental basis that underpins it (refs summing this up), this scaling relationship has become one of the most useful proxies to many elements of biology, including predator-prey interactions \citep{brown2004}. While body size and metabolic rate have featured as the main focus of this drive to find fundamental elements within biological complexity the dimensionality of the arena these interaction play out in has more recently gained attention as key component.


Similar to the formalisation of metabolic theory in the mid 90's the role of habitat complexity in biology has its roots in utilising the mathematics of fractals in order to explain macroecological patterns. \cite{morse1985fractal} was the first to utilise this geometry to describe how the space filling properties of vegetation can determine arthropod community densities through habitat availability. The the effects of habitat dimensionality was further explored in particular in primary consumers and their prey etc (see Pawar and my old thesis for refs). More recently habitat dimensionality has been re-framed as the dimensionality of the interaction between predator and prey \citep{pawar2012dimensionality}. This framework allows dimensionality to be explored across groups and foraging strategies and extend the predictive ability of body mass scaling through the finding that high dimensional interactions have steeper scaling with body mass then expected from standard theory. It also opens up the importance of the role of sensory ecology within such interactions, in particular as such sensory systems have shown different scaling in comparison to that predicted from metabolic theory. %maybe need to include something about that in time perception model.


This thesis draws on how these two fundamental pillars of biological structure affect predator-prey interactions across the range of contexts these interactions take place. By using comparative methods I will focus on three areas; how do species sample and perceive the temporal dimension; the role of habitat dimensionality in prey species life history; and the role of habitat dimensionality in the evolution of predatory traits. By understanding the forces shaping species within the dimensions of their habitats we can gain deeper understanding of the mechanics of predator-prey interactions as a whole.\\


\section{\uppercase{R}esearch outline}


\textbf{Chapter 2: Body size, metabolic rate and visual temporal perception in vertebrates.}


 All organisms must perceive the temporal dimension of their environment. This is particularly true of predators and their prey that need to accurately track and predict their adversaries' motion. Here I collate data on a measure of visual temporal perception called critical flicker fusion to test whether species that are predicted to be more maneuverable can perceive events at finer scales. I show that, as expected, small species with high metabolic rates have the fastest perception of time. This has important consequences for the ability of predators to capture their prey and I discuss some examples of adaptations in predator species that potentially mitigate against this general trend.\\


\textbf{Chapter 3: Ecology and mode-of-life explain lifespan variation in birds and mammals.}


Maximum lifespan varies strongly with body mass yet many species live far longer than expected given their size. This may reflect interspecific variation in extrinsic mortality, as life-history theory predicts investment in long-term survival when extrinsic mortality is reduced. Here, I investigate how ecological and mode-of-life traits that are predicted to reduce extrinsic mortality influence lifespan across mammals and birds. I show that species associated with high dimensional habitats, namely arboreal and volant species, show longer lifespan than expected for their body size. I discuss how habitat dimensionality may affect exposure of prey to predation pressures and the role of other ecological traits including fossoriality, eusociality, and activity patterns.\\




\textbf{Chapter 4: Habitat dimensionality and a diet of eggs; the evolution of venom loss in snakes.}


Despite the obvious advantages of possessing venom there is little explanation for variation in the volume and toxicity of venom in snakes. This is particularly apparent in species that partially or fully lose the capacity to produce venom such as demonstrated in sea snake species. As venom is primarily used for capturing prey I test whether fundamental factors, including habitat dimensionality and diet, affect the amount of venom produced within a species through their influence of encounter rates and prey toxicity resistance. By collating data on venom toxicity (LD50), diets, body size, environment dimensionality and both maximum and minimum venom volumes comprising of over 75 species, I show that species found in high dimensional environments or that have egg-based diets produce less venom than their counterparts. I also demonstrate the prey-specific nature of venom toxicity using phylogenetic distance between diet species and LD50 model species as a test. I discuses the possible mechanisms and the implications of these results in relation to both evolution of venom and costly predatory traits in general.\\

%
%
Finally, in \chapref{discussion}, I close with a discussion of the
limitations of the methods used in the thesis, and suggest some future
directions.

\section{\uppercase{A}dditional work}
In addition to the chapters enclosed in this thesis, I have also been involved in the following research during my studies:\\

\begin{singlespace}
Donohue, I., Petchey, O.L., Montoya, J.M., Jackson, A.L., McNally, L., Viana, M., Healy, K., Lurgi, M., O’Connor, N.E. \& Emmerson, M.C. (2013). On the dimensionality of ecological stability. Ecology Letters, 16(4), 421-429. \\
\end{singlespace}

\noindent
I was involved with the conception, data analysis and write-up of this paper. \\

\begin{singlespace}
Kane, A., Ruxton, G.D., Jackson, A.L. \& Healy, K.(2013). Body size drives importance of scavenging in theropods. In Review in Evolution. \\
\end{singlespace}

\noindent
I was involved with the conception, data collection and analysis, and write-up of this paper. \\



%\bibliographystyle{PLoS-Biology}
%\bibliography{bibfile}

