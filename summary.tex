\chapter*{Summary}
\chaptermark{Summary}
\addcontentsline{toc}{chapter}{Summary}

Predator-prey interactions are an important evolutionary driver and a central component of ecosystem structure. However, the context dependent nature of these interactions, as reflected by the diversity of species involved, means that understanding them at a fundamental level is required for purposeful predictions. In this thesis I explore the role of two fundamental components of predator-prey interactions: habitat dimensionality and body size. I investigate the fundamental role of body size and habitat dimensionality across three chapters that represent stand-alone publications consisting of: the role of body size on the ability of species to perceive the temporal dimension of their environment; the role of habitat dimensionality and other traits relating to predation pressure on life history evolution; and the role of habitat dimensionality on the evolution of venom in snakes. Throughout my thesis I use a comparative approach to show that both habitat dimensionality and body size are key components that determine the mechanics of predator-prey interactions and hence ecological and evolutionary systems as a whole.

%%% Local Variables:
%%% TeX-master: "thesis"
%%% TeX-PDF-mode: t
%%% End:
