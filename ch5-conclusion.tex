\chapter{Discussion}
\label{chap:discussion}


\section{Considering dimensions}

One of the central goals in ecology and evolution is to understand the complexity of the biological world. Several approachs to simplify this problem have included top down approaches such as the metabolic theory of ecology \citep

 Both top down and bottom up approaches have been used to try and understand such systems. However whatever the approach the physics of our world must be included in order for such models to accurately reflect the reality we live in. 

By including this factor changes in understand sensory ecology, life history evolution and the evolution of predatory traits. 

For example by understanding the importnace of temporal ability you get a biogeogrphy thing.

For longevity this will help dentify species which are likley to have solved the abilty to live long. Further more this will help understand the patterns e see?


For snakes this study resolves many of the questions surrounding it and is important for understanding predator-prey interactions.
More stuff is needed to be done.

 

\section{\uppercase{F}uture directions}

This thesis focused mainly detecting patterns of variation realted to differences in habitat structure. Taking these findings further will require the use of more refined data or other thecnuiqes.

One particular useful approach to understand sensory perception is the use of neural netwrk models, perhaps robots etc.


Future direction in the life-history work would likley incude understanding the role of fossorial and the abilty t escape in multiple directions. Here neural network models coupled with agent based approaches may be useful in particular to understand the important of spatial dimensioanlity to the abilty to escape. Further comparitive analyiss would also help in other groups including fossoirailaty in reptiles and other ectoderms. Other data that incudes moreof the complexity of life history may also resolve some of the difficulties in longevity. Such data as the Compdra may help in this in partiuclr in  coupling thertical models of aging with different distrubutions of mortality acorss species life-histories. 


The analysis presented on the evolution of venom in snakes represents an initail atempt of understanign the ecology of venom evoution on a large scale. Despite the realtive pacuaty of data there are clear results. ANother apporach is to use larger evolution binary model or something.



\section{\uppercase }


\bibliographystyle{PLoS-Biology}
\bibliography{bibfile}








