\chapter[Evolution of venom loss in snakes]{Habitat dimensionality and a diet of eggs: The evolution of venom loss in snakes.}
\label{chap:Snake}



%\begin{wrapfigure}{r}{0.5\linewidth}
% \includegraphics[width=.5 \textwidth,right]{ch4-snakes/venom.png}
%\label{fig:fig1}
%\end{wrapfigure}


%\begin{SCfigure}[][!h]
%\centering

% \caption*{``Round and round they went with their snakes, snakily...''\\
%~\\ %cant get Aldous Huxleyto move to the side. Brave New World
%  {Aldous Huxley}}

%  \includegraphics[width=0.45\textwidth]%
%    {ch4-snakes/venom.png}% picture filename

%  \label{ }

%\end{SCfigure}



%\begin{wrapfigure}{c}{20cm}{h}
    \begin{minipage}{0.45\textwidth}
    %\begin{SCfigure}
      \centering
      {~\\
      ``Round and round they went with their snakes, snakily...''\\
%~\\ %cant get Aldous Huxleyto move to the side. Brave New World
  {Aldous Huxley}}
    \end{minipage}%
    \begin{minipage}{0.5\textwidth}
            \includegraphics[width=1.0\textwidth]{ch4-snakes/venom.png}
%  \end{SCfigure}
    \end{minipage}
%\end{wrapfigure}



%\begin{figure}[!h]  
%caption{``Round and round they went with their snakes, snakily...''}
%  \includegraphics[width=.4 \textwidth, right]{ch4-snakes/venom.png}
%\end{figure}



\begin{abstract}

The evolution of venom in snakes offers a distinctive system of predator-prey interactions of ecological, evolutionary and medical interest. Despite this many fundamental questions regarding the evolution of this trait still persist, such as the perplexing range in the toxicity and quantities of venom found across species. For example, despite the obvious benefits of possessing venom, many species have either entirely lost or have extremely low venom volumes and toxicities. One of the potential answers to this paradox are the costs associated with replenishing large amounts of venom, in particular in species where venom is surplus to the requirement of capturing prey. Here I test several hypotheses relating to the ecological factors that may be associated with the atrophy of venom in snakes including diet, body size and habitat dimensionality. Further to this I test the co-evolutionary relationship between venom toxicity and venom volume along with the levels of prey-specificity in a comparative analysis using 101 species of venomous snakes. I find that species associated with high dimensional habitats and a diet of eggs show atrophy of venom through reduced venom volumes and in the case of ovivory, reduced toxicity. These results also show that, despite predictions, levels of toxicity show no association with the volume of venom produced by a species. However snake venom was found to be prey-specific with venoms tested on prey models closely related to the diet of the focal species showing higher levels of toxicity. Overall snake venom provides a remarkable story of predator-prey interactions with a rapidly shifting arms race dynamic. Understanding venom evolution is hence both important to our understanding of predator trait evolution as a whole and the ecology of this important predatory group.


\end{abstract}

\section{Introduction}

The evolution of predatory traits can often be the defining feature of whole clades of animals, from the evolution of the first jawed fish to the use of silk for web construction in arachnids. Perhaps nowhere is this better illustrated than in the evolution of venom in snakes. Through the processes of the reduction and eventual loss of limbs beginning over 160 million years ago \citep{caldwell2015oldest}, snakes have relied on extreme adaptations in order to capture and kill prey items such as using complex venom cocktails \citep{casewell2013complex,fry2012structural}. This reliance on venom as a primary means to capture prey in many species of snakes makes them an excellent group to study how fundamental aspects of ecology, such as habitat dimensionality, can drive the evolution of costly predatory traits and predator-prey interactions in general.


Snake venoms rely on a complex of proteins and other compounds to create neurotoxic and hemotoxic affects in their target prey \citep{greene1997snakes,casewell2013complex}. While the biological properties these mixtures possess has led to a wide body of biomedical research, there is surprisingly little understanding of the ecological pressures associated with the evolution of venoms in snakes \citep{greene1997snakes,casewell2013complex}. Functionally, venoms are used for both foraging and anti-predator defence. However anti-predatory defence is likely to be of secondary functionality of venoms, as reflected by the lack of correlation between lifespan and the possession of venom in snakes \citep{hossie2013species}, with a role in prey acquisition a more likely primary function \citep{casewell2013complex}. However, despite the clear functionality and benefits associated with possessing venom, many species of snakes have paradoxically lost this ability secondarily. Such atrophy of venom is seen at its most extreme vestigial state in the marbled sea snake \textit{Aipysurus eydouxii}, which despite being a member of one of the most venoms groups of snakes, is incapable of even delivering the residual amounts of venom still found in its vestigial glands \citep{li2005eggs}. While not all snakes show such extreme levels of atrophy, intermediate levels of reduced venom volumes and toxicities can be found across diverse groups of snakes \citep{fry2012structural}. One possible solution to this paradox is the investment cost associated with venom replenishment.


Venom synthesis carries a cost both with regards to the energy requirements of venom production and the lost opportunities while replenishing such venom. After venom extraction pit-vipers have been shown to greatly increase their metabolic rates over a 72 hour period \cite{mccue2006cost}. Also, although other species show lower metabolic costs associated with producing venom \citep{pintor2010costs}, species that are heavily dependent on their venom to capture prey would also suffer lost opportunities over the replenishment period \citep{young2002snakes}. For example \textit{Bitis arietans} requires at least seven days to fully restock its venom glands \citep{currier2012unusual}. This cost of venom replenishment is also reflected in the behaviours of some rattlesnake species that meter the amount of venom they inject into their prey species \citep{hayes1995venom} and also through the occurrence of dry bites associated with defensive behaviour \citep{morgenstern2013venom}. This cost of venom may explain the case of \textit{Aipysurus eydouxii}, where a switch to an egg-based diet is likely to be the driver of venom loss due to the removal of the need to incapacitate their prey \citep{li2005eggs}. However many species do include eggs within their diet \citep{de2006historical} yet still show considerable variation in the toxicity and volume of venom they produce pointing towards other ecological or evolutionary drivers of venom evolution.


One key component in the evolution of venom is the identity of the target prey species. The venoms of several snake species show a strong prey-specific element with regards to the toxicity. For example \cite{barlow2009coevolution} showed that within \textit{Echis} viper populations that feed on different prey groups, individuals show higher toxicities for their preferred prey items. Similarly Malaysian pit-vipers show variation in snake venoms that correlate with diet \citep{daltry1996diet}. However, the presence of such prey specificity in venom is not clear in all snakes species, making the ubiquity of this interaction unclear \citep{williams1988variation}. One possible explanation for this is the arms race between the evolution of venom potency in snakes and the corresponding evolution of resistance in species under heavy predation. For example ground squirrels show resistance to the venom of their rattlesnake predators \citep{poran1987resistance}, and there is evidence that resistance to pit viper venoms in opossums has led to a switch in predator-prey roles with opossums now the predators of pit vipers \citep{voss2013opossums}. This arms race has resulted in the rapid evolution of genes associated with venom as demonstrated in the king cobra genome \citep{vonk2013king}, however a simpler evolutionary response to such resistance may be to increase the amount of venom a species produces.


The evolution of snake toxicity depends on rapid genomic evolution, such as through gene duplication, in order to keep pace with prey resistance \citep{vonk2013king}. Such evolutionary events are likely to be relatively rare \citep{vonk2013king}, which may result in species relying on increased venom doses in order to overcome prey resistance in the short term. Species with low toxicities would hence be expected to compensate by producing larger reservoirs of venom with which to increase dosages to prey, while species with high toxicities would be expected to reduce venom production in order to offset unnecessary costs \citep{mccue2006cost}. Such variation in prey resistance may also lead to "overkill" type behaviours were individuals inoculate doses far in excess of that required to incapacitate prey items \citep{sasa1999diet,mebs2001toxicity}. This overkill behaviour is also likely to in part be an artefact to the common use of mice to determine venom toxicities instead of a snake's natural prey \citep{da2001prey}. A more realistic interpretation of the data requires including the physiological distance such test species have in comparison to the snake's diet, with species that have a diet close to that of the test model expected to show higher toxicities indicative of adaptive prey specific venoms. While the evolution of venom is likely to be strongly influenced by the arms race between predator and prey, expected encounter rates with such prey may also be an important determinant in the volume and toxicity of a snake's venom.


While incapacitating prey is a primary function of venom, venom evolution might also be expected to be influenced by the probability of encountering prey. One such aspect which may influence these probabilities is habitat dimensionality. As encounter rates show higher scaling with body mass in high dimensional interactions, such as expected in arboreal and aquatic snakes, investment in venom production may be decreased due to the higher probability of encountering prey which would reduce the costs associated with failed predation attempts. High dimensionality may also reduce venom volumes in order to increase replenishment rates which would allow species to fully exploit such increased encounter rates. In particular arboreal species show faster digestion rates than terrestrial species hence increased replenishing rates may be strongly selected in these species \citep{lillywhite2002patterns}. Conversely high habitat dimensionality may also increase the venom volume produced through the increased capacity of prey species to escape in multiple directions \citep{healy2014ecology,moller2010up}.
%is this reference a bit cheecky

Here I use a comparative approach in order to test multiple hypothesis of venom evolution in snakes. Using data on venom toxicity, species venom volume capacity, size, diet and environment I test the relationship between venom toxicity and both the maximum and mean volume of venom available snake species. I also test whether species that have diets phylogenetically similar to the test animal used to determine toxicity show increased toxicities and whether species that include eggs within their diet show atrophy of both venom toxicity and volume. Finally I test whether species that inhabit high dimensional environments, namely arboreal and aquatic species, show different levels of venom volume and toxicity.

\section{Methods}
\subsection{Data}

As a measure of venom toxicity I used median lethal dose (LD$_{50}$), the individual dose required to kill 50\% of a population of test animals, where lower values of LD$_{50}$ indicate a higher venom toxicity \citep{chippaux1991snake}. As the route of inoculation can affect LD$_{50}$ \citep{chippaux1991snake} only values estimated from either intravenous, subcutaneous, intrapulmonary or intramuscular inoculation routes were used, with a fixed term included to account for the variation between these routes. While most studies determine LD$_{50}$ values using mouse test animals I also included studies that used alternative models as snake venom potency is likely to be linked to diet \citep{barlow2009coevolution}. I used both reported maximum and mean dry weight (mg) as a measure of venom volume as the dry weight represents the active proteinaceous component of venom and is likely to represent the most costly component to produce. In the case of multiple estimates from different studies the mean values across the studies were used as the value for that species with the maximum across all studies used as the overall maximum value. When available, data on sub species was included as a separate measure as venoms can show large variations across sub species \citep{chippaux1991snake}.


Diet data was collated from the literature using studies with quantitative estimates of prey proportions, mainly from studies of stomach contents. As prey items were rarely identified to lower taxonomic levels diet was categorized as in \cite{allen2013evolution} into six prey categories; invertebrates, fish, amphibians, lizards, birds and mammals. A separate term indicating the inclusion of eggs within a species diet was also included.


To test whether snakes with prey phylogenetically close to the LD$_{50}$ test species had higher toxicities I calculated a score of the phylogenetic distance between the LD$_{50}$ test species and the groups present in the snakes diet. This was calculated as the sum of the phylogenetic distance, using average estimates from TimeTree \citep{hedges2006timetree}, between each prey group and the LD$_{50}$ model by the proportion of each prey group reported in each snake species diet. For example a species with a diet of 20\% mammals, 50\% fish and 30\% reptiles with a LD$_{50}$ measured using mice would have a score of 0.2\*(0) + 0.5\*(400.1) + 0.3\*(296) = 288.85.


Species habitat was categorised as either terrestrial, fossorial, aquatic or arboreal based on literature accounts. In order to directly test any effect of the dimensionality of habitat environment each environment was scored, as in \cite{pawar2012dimensionality}, with terrestrial and fossorial environments scored as two-dimensional and arboreal and aquatic scored as three-dimensional.
As venom volume is known to increase with body size \citep{mirtschin2002influences}, it was included in the analysis using total length values from the literature, primarily from the compilation of \citep{boback2003empirical} and from field guides and other works on regional snake faunas. To allow direct comparison with other allometric scaling studies, body length was converted into mass using the conversion $Mass (g) = 0.00035({Total\,Length(cm)})^{3.02}$ from \cite{pough1980advantages}.


Mass, LD$_{50}$, venom volume and phylogenetic distance between diet and model were log$_{10}$ transformed, mean centred and expressed in units of standard deviation prior to analysis. Significance was determined for the fixed effects when 95\% of the data is greater or less than 0. To correct for phylogeny I used the tree from (\cite{pyron2014early}, Figure 4.1). 


\begin{figure}[h]
  \centering
  \includegraphics[width=.95\textwidth]{ch4-snakes/fig1_snake_phylo.pdf}%Note that this is the path for the folder
  %for chapter 2 that has the Tastes_funny.jpg file within it.
  \caption[ ]{Phylogeny of species used in analysis using \cite{pyron2014early}. Scale bar represents million of years. Species coloured blue are aquatic, species coloured green are arboreal and species coloured black are terrestrial or fossorial.}
  \label{fig:Figure 4.1.}
\end{figure}


Overall, data was collated for 101 species (76 species with maximum venom volume estimates, 99 species with average venom volume estimates).


\subsection{Analyses}

To test these hypotheses I fit multiple response phylogenetic mixed models using the MCMCglmm package \citep{hadfield2010mcmc} in R 2.14.2 \citep{RCran}. As venom volume and LD$_{50}$ are likely to have co-evolved, both were included as response variables (allowing for covariance between them to be estimated) with mass, LD$_{50}$ inoculation method, habitat dimensionality, the presence of eggs in diet and phylogenetic distance from LD$_{50}$ model included as explanatory variables. I fit two separate models; one using maximum venom volume and one with average venom volume. Phylogeny was controlled by including it as a random term using the MCMCglmm package \citep{hadfield2010mcmc}. Variation due to multiple measures on individual species, mostly to allow the inclusion of separate values for sub-species, was included using a separate random term at the species level. All models were fitted with uninformative priors by using inverse-Wishart parameter expanded priors \citep{hadfield2010mcmc} with burn-in, thinning and number of iterations determined to ensure effective sample sizes which exceeded 1000 for all parameter estimates and convergence tested using the Gelman-Rubin statistic \citep{gelman1992inference}. 

\section{Results}

After controlling for phylogeny these analyses showed that smaller species that inhabit high dimensional environments have both lower mean and maximum volumes of venom (Figures 4.2 and 4.3, Tables 4.2 and 4.5). Toxicity was affected by the route of inoculation, with intravenous and intrapulmonary inoculation routes showing lower LD$_{50}$ in comparison to subcutaneous measures, and the phylogenetic distance between the LD$_{50}$ model and species diet, with diets closer to the LD$_{50}$ model showing higher toxicities (Tables 4.3 and 4.6). Species with egg based diets showed a significant reduction in toxicity and maximum venom volume but not average venom volume (Tables 4.2, 4.3, 4.5, 4.6).




\begin{figure}[h!]
  \centering
  \includegraphics[width=.95\textwidth]{ch4-snakes/figure2aver.pdf}%Note that this is the path for the folder
  %for chapter 2 that has the Tastes_funny.jpg file within it.
  \caption[ ]{Log$_{10}$ Body size (g) against Log$_{10}$ mean venom volume (mg). Red points and line indicate species associated with low dimensional habitats. Blue points and line indicate species associated with high dimensional habitats. Triangles represent species with eggs found in their diet.}
  \label{fig:Figure 4.2.}
\end{figure}




\begin{figure}[h!]
  \centering
  \includegraphics[width=.95\textwidth]{ch4-snakes/figure2max.pdf}%Note that this is the path for the folder
  %for chapter 2 that has the Tastes_funny.jpg file within it.
  \caption[ ]{Log$_{10}$ Body size (g) against Log$_{10}$ maximum venom volume (mg). Red points and line indicate species associated with low dimensional habitats. Blue points and line indicate species associated with high dimensional habitats. Triangles represent species with eggs found in their diet.}
  \label{fig:Figure 4.3.}
\end{figure}


Their was no correlation between LD$_{50}$ and either maximum or mean venom volumes (Tables 4.1 and 4.4). Both LD$_{50}$  and venom volume show high phylogenetic variance with phylogeny showing a higher association with LD$_{50}$ in comparison to venom volume in both models (Tables 4.1 and 4.4). 

\clearpage

\begin{table}[H]
  \centering
    \caption[ ]{Variance co-variance structure of random terms in the model using mean venom volumes. The variance associated with mean volume and LD$_{50}$ and the covariance between mean volume and LD$_{50}$ is given for the phylogenetic structure (Phylogeny), the residuals and a term to account for variation associated at the species and subspecies level (Species). Lower CI = lower 95\% credibility interval, Upper CI = Upper 95\% credibility interval.}
\begin{tabular}{*5l}    \toprule
\emph{Terms} & \emph{Estimate} & \emph{Lower CI} & \emph{Upper CI}\\\midrule
\textbf{Phylogeny} &   &   &  \\ 
Variance: Mean volume & 0.456 & 0.145 & 0.847 \\
Covariance: Mean volume and LD$_{50}$ &-0.003  &-0.290  & 0.301 \\
Variance: LD$_{50}$ & 0.909 & 0.479 & 1.452 \\

 &   &   &  \\

\textbf{Residuals} &   &   &  \\ 
Variance: Mean volume & 0.009 & 0.007 & 0.011 \\
Covariance: Mean volume and LD$_{50}$ & 0.003  &  -0.004  & 0.011 \\
Variance: LD$_{50}$ & 0.268 & 0.215 & 0.328 \\

 &   &   &  \\ 

\textbf{Species} &   &   &  \\ 
Variance: Mean volume & 0.308 & 0.156 & 0.462 \\
Covariance: Mean volume and LD$_{50}$ & 0.030  &  -0.040  & 0.118 \\
Variance: LD$_{50}$ & 0.055 & 0.001 & 0.170 \\\bottomrule
 \hline
\end{tabular}
  \label{tbl:Table 4.1.}
\end{table}




\begin{table}[H]
  \centering
    \caption[ ]{Relationship between average venom volume and body mass, inoculation method, habitat dimensionality, presence of eggs in diet and average phylogentic distance between diet and LD$_{50}$ model. Lower CI = lower 95\% credibility interval, Upper CI = Upper 95\% credibility interval.}
\begin{tabular}{*5l}    \toprule
\emph{Fixed Terms} & \emph{Estimate} & \emph{Lower CI} & \emph{Upper CI}\\\midrule
\textbf{Intercept} & 0.200  &  -0.161 & 0.567 \\ 
\textbf{Body Mass} & 0.510  & 0.442 & 0.564 \\ 
\textbf{Inoculation Route} &  &  &  \\ 
 Intravenous (IV) &  -0.011 &  -0.052 & 0.030 \\
 Intrapulmonary (IP) &  -0.010 &  -0.052 & 0.030 \\ 
 Intramuscular (IM) &  -0.009 &  -0.056 & 0.042 \\
  &  &  &  \\ 
\textbf{Dimension 3D} &  -0.829 &  -1.286 &  -0.396 \\ 
\textbf{Eggs in diet} &  -0.741 &  -1.325 &  -0.206 \\ 
\textbf{Phylogenetic disparity of diet to model} &  -0.003 &  -0.029 & 0.019 \\\bottomrule
 \hline
\end{tabular}
  \label{tbl:Table 4.2.}
\end{table}



\begin{table}[H]
  \centering
    \caption[ ]{Relationship between LD$_{50}$ and body mass, inoculation method, habitat dimensionality, presence of eggs in diet and average phylogenetic distance between diet and LD$_{50}$ model in average venom volume model. Lower CI = lower 95\% credibility interval, Upper CI = Upper 95\% credibility interval.}
\begin{tabular}{*5l}    \toprule
\emph{Fixed Terms} & \emph{Estimate} & \emph{Lower CI} & \emph{Upper CI}\\\midrule
\textbf{Intercept} & 0.200  &  -0.161 & 0.567 \\ 
\textbf{Body Mass} & 0.134  & 0.016 & 0.262 \\ 
\textbf{Inoculation Route} &  &  &  \\ 
 Intravenous (IV) &  -0.624 &  -0.842 &  -0.435 \\
 Intrapulmonary (IP) &  -0.537 &  -0.746 &  -0.309 \\ 
 Intramuscular (IM) &  -0.228 &  -0.455 & 0.049 \\
  &  &  &  \\ 
\textbf{Dimension 3D} &  -0.202 &  -0.670 & 0.243 \\ 
\textbf{Eggs in diet} & 0.457 &  -0.187 & 1.065 \\ 
\textbf{Phylogenetic disparity of diet to model} & 0.360 & 0.248 & 0.463 \\\bottomrule
 \hline
\end{tabular}
  \label{tbl:Table 4.3.}
\end{table}





%%%%%new set of tables

\begin{table}[H]
  \centering
    \caption[ ]{Variance co-variance structure of random terms in the model using maximum venom volumes. The variance associated with mean volume and LD$_{50}$ and the covariance between mean volume and LD$_{50}$ is given for the phylogenetic structure (Phylogeny), the residuals and a term to account for variation associated at the species and subspecies level (Species). Lower CI = lower 95\% credibility interval, Upper CI = Upper 95\% credibility interval.}
\begin{tabular}{*5l}    \toprule
\emph{Terms} & \emph{Estimate} & \emph{Lower CI} & \emph{Upper CI}\\\midrule
\textbf{Phylogeny} &   &   &  \\ 
Variance: Maximum volume & 0.500 & 0.156 & 0.960 \\
Covariance: Maximum volume and LD$_{50}$ & 0.156  &  -0.181  & 0.496 \\
Variance: LD$_{50}$ & 0.901 & 0.349 & 1.477 \\

 &   &   &  \\

\textbf{Residuals} &   &   &  \\ 
Variance: Maximum volume & 0.003 & 0.002 & 0.004 \\
Covariance: Maximum volume and LD$_{50}$ & 0.006  & 0.001  & 0.012 \\
Variance: LD$_{50}$ & 0.275 & 0.001 & 0.361 \\

 &   &   &  \\ 

\textbf{Species} &   &   &  \\ 
Variance: Maximum volume & 0.230 & 0.103 & 0.373 \\
Covariance: Maximum volume and LD$_{50}$ &  -0.036  &  -0.141  & 0.052 \\
Variance: LD$_{50}$ & 0.061 & 0.001 & 0.188 \\\bottomrule
 \hline
\end{tabular}
  \label{tbl:Table 4.4.}
\end{table}


\begin{table}[H]
  \centering
    \caption[ ]{Relationship between maximum venom volume and body mass, inoculation method, habitat dimensionality, presence of eggs in diet and average phylogenetic distance between diet and LD$_{50}$ model. Lower CI = lower 95\% credibility interval, Upper CI = Upper 95\% credibility interval. }
\begin{tabular}{*5l}    \toprule
\emph{Fixed Terms} & \emph{Estimate} & \emph{Lower CI} & \emph{Upper CI}\\\midrule
\textbf{Intercept} & 0.043  &  -0.339 & 0.379 \\ 
\textbf{Body Mass} & 0.757  & 0.708 & 0.797 \\ 
\textbf{Inoculation Route} &  &  &  \\ 
 Intravenous (IV) &  -0.006 &  -0.031 & 0.019 \\
 Intrapulmonary (IP) & 0.008 &  -0.019 & 0.037 \\ 
 Intramuscular (IM) & 0.003 &  -0.037 & 0.047 \\
  &  &  &  \\ 
\textbf{Dimension 3D} &  -1.212 &  -1.638 &  -0.763 \\ 
\textbf{Eggs in diet} &  -0.564 &  -1.219 & 0.063 \\ 
\textbf{Phylogenetic disparity of diet to model} &  -0.001 &  -0.032 & 0.033 \\\bottomrule
 \hline
\end{tabular}
  \label{tbl:Table 4.5.}
\end{table}



\begin{table}[H]
  \centering
    \caption[ ]{Relationship between LD$_{50}$ and body mass, inoculation method, habitat dimensionality, presence of eggs in diet and average phylogenetic distance between diet and LD$_{50}$ model in maximum venom volume model. Lower CI = lower 95\% credibility interval, Upper CI = Upper 95\% credibility interval.}
\begin{tabular}{*5l}    \toprule
\emph{Fixed Terms} & \emph{Estimate} & \emph{Lower CI} & \emph{Upper CI}\\\midrule
\textbf{Intercept} & 0.043  &  -0.339 & 0.379 \\ 
\textbf{Body Mass} & 0.145  &  -0.018 & 0.305 \\ 
\textbf{Inoculation Route} &  &  &  \\ 
 Intravenous (IV) &  -0.656 &  -0.892 &  -0.440 \\
 Intrapulmonary (IP) &  -0.561 &  -0.804 &  -0.326 \\ 
 Intramuscular (IM) &  -0.191 &  -0.531 & 0.152 \\
  &  &  &  \\ 
\textbf{Dimension 3D} & 0.240 &  -0.300 & 0.836 \\ 
\textbf{Eggs in diet} & 0.705 & 0.076 & 1.510 \\ 
\textbf{Phylogenetic disparity of diet to model} & 0.251 & 0.080 & 0.436 \\\bottomrule
 \hline
\end{tabular}
  \label{tbl:Table 4.6.}
\end{table}




\section{Discussion}


These results show that, after controlling for body size and phylogeny, the evolution of venom in snakes is linked to their habitat, prey and the costs associated with producing large amounts of venom. In particular the atrophy of venom seen across venomous snakes is shown to be linked with arboreal and aquatic lifestyles, through a reduction of venom volumes,  and, in the case of maximum venom volume, the presence of eggs in their diet. This is likely due to the costs of producing venom \citep{pintor2010costs} outweighing the benefits associated with maintaining it. 


One of the drivers behind such a reduced benefit of venom production for species in three dimensional environments may be the expected increased probability of encountering prey items in such environments \citep{pawar2012dimensionality}. This would result in species being under weaker selection to maintain the larger reservoirs of venom required to respond to the rarer event of encountering a prey item in a low dimensional environment. The biomechanical limitations of arboreal lifestyles is also suggested to lead to faster digestion rates in these species \citep{lillywhite2002patterns}, which in turn may lead to smaller venom volumes which would facilitate faster venom replenishment rates. The limitations affecting the size of predators in these environments would also restrict prey body size in arboreal species. Likewise marine snakes are also known to feed disproportionately on smaller fish then expected \citep{voris1981size}. Since some species are known to meter venom depending on prey size \citep{hayes1995venom}, it may be the size distribution of prey available in these habitats that influences the evolution of venom volume. Hence although the patterns of venom volumes are clearly related to habitat dimensionality, and hence interaction dimensionality, the mechanism behind this pattern requires further investigation. %could include the sum.v argument that even though 3D species are more specalised this doesnt affect the results. 


The nature of a snake's diet was also found to affect venom toxicity and to a lesser extent venom volume. Species with even small proportions of eggs in their diets show both reduced maximum venom volumes and lower venom toxicities. This is unsurprising as the benefits of venom to an ovivorous diet are likely to be low \citep{li2005eggs}. This atrophy in ovivorous snakes also reaffirms the primary foraging function of venom with any digestive \citep{rodriguez1992venom} or defensive functions \citep{jansa2011adaptive} more likely to represent secondary benefits. However these results should be interpreted with caution due to the low number of ovivorous species within the analysis (eight with only four having diets consisting of greater then 20\% eggs) with further data required to fully gauge the evolutionary role of egg eating in snakes. 


Apart from the reduction of the venom apparatus in species which show a shift from carnivory to ovivory \citep{li2005eggs}, these results also show that snake venom toxicity is prey-specific. While prey specificity has been shown within particular groups of snakes \citep{barlow2009coevolution,richards2012venom,daltry1996diet} this is the first study to show venom prey specificity across all venomous snakes. This result of increased toxicity with reduced phylogenetic distance between diet LD$_{50}$ model suggests that while their are several cases of prey species developing resistance to venom \citep{lillywhite2002patterns}, snakes in general are "ahead" in the arms race between prey venom resistance and predator venom toxicity.


While these results further demonstrate the arms race between venom evolution and prey resistance, they also surprisingly show no evidence of co-evolution between venom volume and venom toxicity evolution. It would be expected that species with low toxicities may evolve compensatory mechanisms such as increased venom volumes to allow them to overcome prey resistance. While the historical lack of appropriate model species for calculating LD$_{50}$ may account for the underestimation of toxicity levels in some species \citep{da2001prey}, even after accounting for such species mismatching here there is no evidence of a correlation between these two aspects of venom functionality. This lack of compensatory evolution may instead be explained by changes in behaviour with species with low toxicities combining the use of venom along with prey holding behaviour or constriction in order to incapacitate prey \citep{shine1985prey}.


While volume and LD$_{50}$ show no co-variance both traits do show strong phylogenetic effects suggesting evolutionary constraints also partially explain the variation in venom lethality and volume across snakes. As expected, LD$_{50}$ evolution shows a higher constraint in comparison to venom volume evolution. This is likely to be a reflection of the requirement for major genetic changes, such as gene duplication events, in order to increase venom toxicity \citep{vonk2013king}. While venom volume shows less phylogenetic autocorrelation, the physiological requirements necessary to house large volumes of venom is likely to be one of limitation in many species, in particular in rear-fanged groups \citep{kardong1982evolution}. However many rear-fanged snakes such as species of \textit{Dispholidus} can contain venom volumes similar or in excess of many front-fanged species showing that this trait is not insurmountable in these groups \citep{kochva1980venom,fry2008evolution}.


Overall this study shows that ecological factors associated with predator-prey interactions are important drivers of venom evolution. While further studies are required to understand the complex nature of such predatory trait evolution these results show that fundamental aspects of predator-prey interactions including size, the dimensionality of their interaction with prey can help understand one of the most medically important and iconic predatory traits, venom.



%\bibliographystyle{PLoS-Biology}
%\bibliography{bibfile}

