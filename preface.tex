\chapter*{Preface} %the * removes the numbers
\addcontentsline{toc}{chapter}{Preface}

Several chapters from my thesis have been published elsewhere:

\lettherebespace
\textsc{chapter one} has been previously published as:
%
\begin{previouspaper}
  \textbf{Healy K.,} McNally, L., Ruxton, G.D., Cooper, N., Jackson, A.L. 2013. Metabolic rate and body size are linked with perception of temporal information. \textit{Animal Behaviour}, 86, 685-696.
\end{previouspaper}
 
As lead author I was involved with the initial conception of the paper, I collected the data, designed and ran the analysis and wrote the manuscript.


\lettherebespace
\lettherebespace
\textsc{chapter two} has been previously published as:
%
\begin{previouspaper}
  \textbf{Healy K.,} Guillerme, T., Finlay, S., Kane, A., Kelly, S.B.A., McClean, D., Kelly, D.J., Donohue, I. Jackson, A.L., Cooper, N. 2014. Ecology and mode-of-life explain lifespan variation in birds and mammals. \textit{Proc. R. Soc. B}, 281, 20140298 .
\end{previouspaper}

As lead author I was involved with the initial conception of the paper, collection and management of the data, designing and running the analysis and writing the manuscript.

\begin{previouspaper}
  \textbf{Healy K.} 2015. Eusociality but not fossoriality drives longevity in small mammals. \textit{Proc. R. Soc. B}, 282, 20142917.
\end{previouspaper}

I replied to the comment by \cite{williams2015ecology} on the paper above were they show eusociality is an important factor to consider for lifespan evolution in fossorial mammals. I carried out additional analysis showing that including the error associated with phylogeny construction further strengthens the association between eusociality and increased lifespan.


%Here I explain what I did for this paper, and what the other coauthors did. Can just repeat this for each chapter if all are published already or in prep. Basically for anything you do with other people.
%%%%%%%%%%%%%%%%%%%%%%%%%%%%%%%%%%%%%%%%%%%%%