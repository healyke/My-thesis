\chapter{Discussion}
\label{chap:discussion}


\section{Considering dimensions}

One of the central goals in ecology and evolution is to understand the complexity of the biological world. There has been several approaches towards simplifying this complexity including the top down approaches of the metabolic theory of ecology \citep{brown2004}, the bottom up approach of dynamic energy budget theory \citep{kooijman2009dynamic} and the entropy based approach of MaxEnt \citep{harte2008maximum}. However whatever the approach the physics of our world must be included in order for such models to accurately reflect the reality of nature. 


Here I have attempted here to make the case for the further inclusion of perhaps one of the most fundamental elements of physical reality, its dimensions. While all life is embedded within the three spatial and one temporal dimension of the universe, how species exploit or interact within these dimensions can determine the nature of many biological interactions. This is reflected in the results of the above analysis which demonstrate the important influence of the ability of a species to exploit these dimensions which can in turn define these interactions. 


While this thesis focuses on how traits associated with predator-prey interactions (spotting, escaping or capturing) are affected by environment dimensionality, these effects are likely to hold important consequences for larger ecological systems in particular through the the strong influence of predator-prey interacts. For example trophic interaction strengths have important influences on ecosystem stability hence systems large bosy size ranges are (read up on this a bit more).


While these factors are likely to be important across biological systems many of the questions outlined within this thesis were restricted by limitations in data availability. While large data sources of data related to MET are now available other data relating to fundamental aspects of ecology are still in the early stages accumulation. For example data on species behaviour, diets, sensory capabilities and other morphologocal trts are still often defined in relativly reductionist ways. Even considering this restriction phylogenetic comparative methods are only likley to . These limitations highlight the large body of reserch to be done in this area of ecology pointing towards a field with good future prospects.

\section{\uppercase{F}uture directions}

This thesis focused mainly detecting patterns of variation realted to differences in habitat structure. Taking these findings further will require the use of more refined data or other thecnuiqes.

One particular useful approach to understand sensory perception is the use of neural netwrk models, perhaps robots etc.


Future direction in the life-history work would likley incude understanding the role of fossorial and the abilty t escape in multiple directions. Here neural network models coupled with agent based approaches may be useful in particular to understand the important of spatial dimensioanlity to the abilty to escape. Further comparitive analyiss would also help in other groups including fossoirailaty in reptiles and other ectoderms. Other data that incudes moreof the complexity of life history may also resolve some of the difficulties in longevity. Such data as the Compdra may help in this in partiuclr in  coupling thertical models of aging with different distrubutions of mortality acorss species life-histories. 


The analysis presented on the evolution of venom in snakes represents an initail atempt of understanign the ecology of venom evoution on a large scale. Despite the realtive pacuaty of data there are clear results. ANother apporach is to use larger evolution binary model or something.





\bibliographystyle{PLoS-Biology}
\bibliography{bibfile}








