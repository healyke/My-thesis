\chapter{Discussion}
\label{chap:discussion}


\section{Considering dimensionality}

One of the central goals in ecology and evolution is to understand the complexity of the biological world. There have been several approaches towards simplifying this complexity including the top down approaches of the metabolic theory of ecology \citep{brown2004} and the bottom up approach of dynamic energy budget theory \citep{kooijman2009dynamic}. Whatever the approach, certain elements of our world must be included in such macroevolutionary approaches in order for these models to accurately reflect the reality of nature. 


Throughout this thesis I demonstrate the importance of including perhaps one of the most fundamental elements of physical reality, its dimensions. While all life is embedded within the three spatial and one temporal dimension of the universe, how species exploit these dimensions can determine the nature of many biological interactions. This is reflected in the results of chapters 2-4 which demonstrate the important influence of the ability of a species to exploit these dimensions affecting how they see the world, how long they live in it and how they make this living. 


While this thesis focuses on how traits associated with predator-prey interactions (targeting, escaping or capturing) are affected by interaction dimensionality, these effects are likely to hold important consequences for larger ecological systems. For example, trophic interaction strengths are one of the main determinants of ecosystem stability \citep{may1972will,pimm1984complexity} with systems with different ratios of interaction dimensionalities likely to demonstrate different dynamic behaviour in response to stress and perturbation \citep{donohue2013dimensionality}. Likewise by including the dimensions over which species interact, a better understanding of the diversity of species form may also be achieved. For example, by acknowledging the importance of prey tracking in mesopelagic systems, several species with unique physiologies to increase temporal perception have been discovered \citep{fritsches2005warm,frank2012light,landgren2014visual} including the first case of true endothermy in a fish \citep{Wegner15052015}.


Throughout this thesis, comparative analyses provided the main tool to investigate the large scale macroecological patterns central to each chapter. However additional methods will be required to further investigate the importance of interaction dimensionality in future work. While each chapter has attempted to resolve some outstanding question within ecology and evolution they in turn raised many more. The future directions associated with these chapters hence include not only consolidating the results found here, but also towards expanding the methodology, data and framework associated with predator-prey interactions.


\section{\uppercase{F}uture directions}

\subsection{Comparative methods}


Throughout this thesis the comparative method employed grew to match the increasing complexity of the hypotheses explored throughout my PhD. From the standard PGLS models in Chapter 2, the inclusion of the error associated with phylogeny construction in Chapter 3 and the multiple response models used in the final chapter, the increased complexity of these models was important to test the questions posed in these chapters. These methods will continue to be useful with regards to questions posed in each of the chapters presented here. 

Since publication of the paper on temporal perception relating to chapter 2 \citep{healy2013metabolic} numerous studies on the sensory perceptual abilities of different species have appeared, in particular in ecological settings such as deep sea environments \citep{kalinoski2014spectral,Wegner15052015,landgren2014visual}, or on the effects such sensory limitations can incur on their behaviour and ecology \citep{bar2015sensory,inger2014potential}. These studies represent growing interest within this field with additional data becoming available opening up the opportunity for more nuanced analysis between species ecologies and their temporal perceptual abilities. In particular paired data on the visual systems between predator and prey species would allow for the testing of the existence of an arms races similar to that seen in snake venom evolution in chapter 4. Other interesting avenues to explore would be the existence of scaling in other sensory systems, including olfactory \citep{uchida2003speed}, auditory \citep{bar2015sensory} and tactile \citep{braam2005touch}. Echolocation is likely to be a particularly fruitful sense to study sensory limitations on predator-prey interactions \citep{bar2015sensory}, in particular through comparison of the use of echolocation in species ranging in size from the smallest mammals (bats and shrews) to the largest predators (sperm whales).


The second chapter of this thesis took advantage of the Bayesian nature of the MCMCglmm comparative approach of \cite{hadfield2010mcmc} in order to include the error associated with constructing a phylogeny. This is likely to prove useful in future approaches, particularly as phylogeny topology is still not fully resolved in many groups \citep{jetz2012global,burleigh2015building,pyron2014early} and as Bayesian generated phylogenies are becoming more available \citep{arnold201010ktrees,jetz2012global}. The importance of using this approach was demonstrated in the difference results obtained with regards to the methods used by \cite{williams2015ecology} and those used in chapter 3 \citep{healy2014ecology,healy2015eusociality}, where including the error associated within the phylogeny showed that eusociality but not fossoriality was the main driver of longevity within the data. 

Irrespective of methodology, these result bring into question the importance of fossoriality in the evolution of longevity. However longevity itself may be causally linked to the evolution of eusociality in mammals, due to the requirement of multiple interactions between individuals over long intervals before the benefits of eusociality can accrue \citep{LukeM}, making this group ill suited for this question. Comparative analyses similar to chapter 3 in groups where the conflict between fossoriality and eusociality does not arise may hence be better placed to resolve the importance of fossoriality in life history evolution. Reptiles and amphibians are two such groups that contain species of varying degrees of fossoriality but are mainly solitary in nature. Another useful future direction using comparative analyses to understand lifespan evolution is to test whether other groups of species that can avail of high dimensional escape routes show similar differences in life-history traits found in chapter 3. Pelagic and benthic marine species may provide a good test case for this hypothesis with benthic species unable to avail of the 3D escape space of pelagic species and as aquatic species would not be subjected to the same size limitations found in arboreal and aerial species.

The final chapter is yet to be published, however the need for further comparative analyses is clear following the results provided above. In particular the inclusion of prey body size data is likely to be important with regards to the affect of habitat dimensionality on snake venom volumes. Also, following my discovery of the association between species that are ovivorous and the atrophy of venom, a larger analysis comparing the ecologies of venomous and completely non venomous species would help extend this finding.


\subsection{Other approaches}

While comparative methods feature as the central method in this thesis, such approaches are generally limited in their scope to identifying large scale macro ecological and evolutionary patterns. For example, due to the lack of appropriate data on sensory ability, the ecological drivers of temporal ability is presently outside the range of such approaches. Likewise the dimensionality of escape space open to a species is often heavily correlated with other aspects of ecological and life-history traits make decoupling such causalities difficult. One such approach which would be particularly beneficial for this question is to link together agent based modelling with neural network modelling. 


Agent based modelling uses simulations of individual "agents" that are defined by simple sets of rules \citep{tisue2004netlogo}. Unlike experimental approaches, agent based modelling allows the full set of parameter space to be explored making this an ideal approach for questions featuring fundamental constraints such as dimensionality. In the case of temporal perception evolution, such rules are relatively simple; predict the future position of a moving target displaying different movement patterns. However while this approach may encapsulate the absolute limits at which temporal perception no longer improves target prediction, the neural and metabolic costs associated with such perceptual abilities, as demonstrated in chapter 2, also needs to be incorporated. Neural networks would provide one such solution by allowing for a more evolutionary approach to optimal temporal perception while also allowing for the ability to test a series of other related questions, including, the optimal temporal perceptions for a series of different prey motion patterns. The neural network approach would also be able to be extended outside of the simulated environment provided by the agent based modelling through the use of robots. 


Robots are essentially an agent based model parameterised within reality, making them an ideal half-way house between simulations and experimental approaches \citep{floreano2010evolution}. Furthermore, while used to study predator-prey interactions such robot systems inadvertently displayed behaviours associated with limitations of their temporal perceptual abilities \citep{floreano2010evolution}. In particular, due to the refresh rates of the cameras used for target tracking and navigation, the robots were found to move only at intermediate speeds despite being capable of much higher speeds. The use of robots to incorporate realistic parameters into such model is also not constrained to terrestrial systems with the aerial robofly \citep{lauder2001aerodynamics} and aquatic robofish \citep{faria2010novel} two examples of extending these approach to other environments.


Such use of simulated environments may also be applicable to further investigations on the importance of the dimensionality of a species escape space. By extending the agent based modelling approach towards incorporating the geometric predator escape models of \cite{howland1974optimal}, the importance of dimensionality can be tested directly within a range of contexts. For example the importance of this escape space could be used to study fish escape strategies \citep{domenici1997kinematics} and also extended into investigating shoaling behaviour in response to shark and whale predation events by incorporating the simple behaviour rules of \cite{couzin2002collective}.


Whatever the approach, future research into how biology fits into and exploits the fundamental aspects of our reality is likely to continue along its current fruitful trajectory. As the most complex entity in the universe it should be no surprise that many of the elements and behaviours of biological systems are still so difficult to predict or understand. By comparing biology to other complex systems, both mathematical and real, we should be able to further delve into the most mysterious element of existence.


%\bibliographystyle{PLoS-Biology}
%\bibliography{bibfile}





