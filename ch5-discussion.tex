\chapter{Discussion}
\label{chap:discussion}


\section{Considering dimensions}

One of the central goals in ecology and evolution is to understand the complexity of the biological world. There has been several approaches towards simplifying this complexity including the top down approaches of the metabolic theory of ecology \citep{brown2004}, the bottom up approach of dynamic energy budget theory \citep{kooijman2009dynamic} and the entropy based approach of MaxEnt \citep{harte2008maximum}. However whatever the approach certain elements of our world must be included in such macroevolutionary approaches in order for such models to accurately reflect the reality of nature. 


Here I have attempted here to make the case for the importance of including perhaps one of the most fundamental elements of physical reality, its dimensions. While all life is embedded within the three spatial and one temporal dimension of the universe, how species exploit or interact within these dimensions can determine the nature of many biological interactions. This is reflected in the results of the above analysis which demonstrate the important influence of the ability of a species to exploit these dimensions affecting how they see the world, how long they live in it and how they make a living. 


While this thesis focuses on how traits associated with predator-prey interactions (spotting, escaping or capturing) are affected by environment dimensionality, these effects are likely to hold important consequences for larger ecological systems in particular through the the strong influence of predator-prey interacts. For example trophic interaction strengths have important influences on ecosystem stability leading to the persistence of certain types of systems. Likewise by including the dimensions over which species interact can help understand the diversity of physiologies found in nature. For example the new endothermic fish. 


While these factors are likely to be important across biological systems many of the questions outlined within this thesis were restricted by limitations in data availability. While large data sources of data related to MET, such as metabolic rates and body sizes, are now available other data relating to fundamental aspects of ecology are still in the early stages accumulation. For example data on species behaviour, diets, sensory capabilities and other morphological traits can be still surprisingly sparse, and when available relatively coarse in their nature. These data however, as demonstrated above, can still provide starting points to test many predicted patterns in nature through comparative analysis. Even with the increase body of data continuously growing and becoming more readily available, such as xxxx, future directions incuding comparative methods on new datasets or separate approaches are likely to offer the best prospects. 
%make this more about these questions requiring more attention of something.


\section{\uppercase{F}uture directions}

\subsection{Comparative methods}


Throughout this thesis the comparative method employed grew to match the increasing complexity of the hypotheses explored throughout my PhD. From standard PGLS models in Chapter one, to the inclusion of the error associated with phylogeny construction in Chapter 2, to the multiple response models used in the final chapter the increased complexity of these models were important to test the questions posed in these chapters. These methods will continue to be useful with regards to questions posed in each of the chapters presented here. 

Since publication of my work on temporal perception \citep{healy2013metabolic} numerous studies on the sensory perceptual abilities of species in particular ecological settings, such as deep sea foraging species fish \citep{kalinoski2014spectral,Wegner15052015,landgren2014visual}, or on the effects such sensory limitations can incur on their behaviour and ecology \citep{bar2015sensory,inger2014potential}. These studies represent the growing interest within this field and the additional data on temporal perception that may allow for more nuanced analysis between species ecologies and their temporal perceptual abilities. Other interesting avenues to explore would be other sensory systems, including olfactory, auditory and tactile. Echolocation is likely to be a particularly fruitful sense to study sensory limitations on predator-prey interactions, in particular through comparison of body size ranges from the smallest mammals (bats and shrews) to the largest predators (sperm whales).


The second chapter of this thesis took advantage of the Bayesian nature of the MCMCglmm comparative approach of \citep{hadfield2010mcmc} in order to include the error associated with constructing a phylogeny. This is likely to prove useful in future approaches as Bayesian generated phylogenies become more available and within groups were phylogeny topology is still debated. In fact the difference of using this approach was shown with regards to the methods used by \cite{williams2015ecology} in response to \cite{healy2014ecology} showing that both eusociality and fossoriality are important to lifespan evolution, whereas the approach used in chapter 2 differed in showing that eusociality but not fossoriality was important. Irrespective of methodology, these result brings into question the importance of fossoriality in the evolution of longevity, however recently other lines of evidence conflict on its importance \citep{faulkes2015molecular}. Large comparative analysis between fossorial and non-fossorial species in groups such as amphibians and reptiles may help to resolve this conflict. Another useful future direction on comparative analysis in lifespan evolution is to test whether species that can avail of high dimensional escape routes show differences in life-history traits. In particular pelagic and benthic marine species are likely to provide a good test case for this hypothesis. 


The final chapter is yet to be published however the need for further comparative analysis is clear following the results provided above. In particular the inclusion of prey body size data is likely to be important with regards to the affect of habitat dimensionality on snake venom volumes. Following the association between species that are ovivorous or in high dimensional habitats and the atrophy of venom seen in the results of chapter 4, a larger analysis comparing ecology of venomous and completely non venomous species. 

 In particular an analysis of the complete absence of venom apparatus within snakes may yield further insights to the apparent atrophy seen in this chapter. As a complete phylogeny of snakes has only recently appeared the opportunity to do this across the whole group has only become achievable. 

\subsection{Other approaches}

While comparative methods feature as the central method in this thesis such approaches are generally limited in their scope to identifying large scale macro ecological and evolutionary patterns. For example due to the lack of appropriate data on sensory ability the ecological drivers of temporal ability is outside of the range of such approaches. Likewise the dimensionality of escape space open to a species is often heavily correlated with other aspects of ecological and life-history traits make decoupling such causalities difficult. One such approach which would be particularly beneficial for this question is to link together agent based modelling with neural network modelling. 


Agent based modelling uses simulations of individual "agents" that are defined by simple sets of rules. Unlike experimental approaches agent based modelling allows the full set of parameter space to be explored making this an ideal approach for questions featuring fundamental constraints such as dimensionality. In the case of temporal perception evolution such rules are relatively simple; predict the future position of a moving target using displaying different movement patterns. However while this approach may encapsulate the absolute limits at which temporal perception no longer improves target predictions the neural and metabolic costs associated with such perceptual abilities, as demonstrated in chapter 2, need to be incorporated. Neural networks would provide one such solution however allowing for a more evolutionary approach to optimal temporal perception while also allowing for the ability to test a series of other related questions including the optimal temporal perceptions for a series of different prey motion patterns. The neural network approach would also be able to be extended outside of the simulated environment provided by the agent based modelling through the use of robots. Robots are essentially an agent based model parametrise within reality, making them an ideal half-way house between simulations and experimental approaches \citep{floreano2010evolution}. Furthermore, while being used to study predator-prey interactions they inadvertently displayed behaviours associated with limitations of their temporal perceptual abilities. In particular due to the refresh rates of the cameras used for target tracking and navigation the robots were found to only move at intermediate speeds despite being capable of much higher speeds.


Such use of simulated environments may also be applicable to further investigations on the importance of the dimensionality of a species escape space. By extending the agent based modelling approach towards incorporating the geometric predator escape models of \cite{howland1974optimal} the importance of dimensionality can be tested directly within a range of contexts. For example the importance of this escape space could be used to study fish escape strategies \citep{domenici1997kinematics} and even extending into the importance of shaoling behaviour in response to shark and whale predations by incorporating the simple behaviour rules of \citep{couzin2002collective}.

These approaches could also be extended outside of the simulated environment provided by the agent based models through the use of robots. Robots are essentially an agent based model parametrise within reality, making them an ideal half-way house between simulations and experimental approaches \citep{floreano2010evolution}. Furthermore, while being used to study predator-prey interactions they inadvertently displayed behaviours associated with limitations of their temporal perceptual abilities. In particular due to the refresh rates of the cameras used for target tracking and navigation the robots were found to only move at intermediate speeds despite being capable of much higher speeds. The use of robots to incorporate realistic parameters into such model is also not constraint to terrestrial systems with aerial (ref) and aquatic robots such as robofish \citep{faria2010novel} also possible ways of testing such idea.


Whatever the approach, future research into how biology fits into and exploits the fundamental aspects of our reality is likely to continue along its current fruitful trajectory. As the most complex entity in the universe it should be no surprise that many of the elements and behaviours are still so difficult to predict or understand. By comparing biology to other complex systems both mathematical and real we should be able to further delve into the most mysterious element of existence.

\bibliographystyle{PLoS-Biology}
\bibliography{bibfile}








